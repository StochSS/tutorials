\chapter{Spatial Stochastic Simulations with StochSS}

%\section{\label{sec:pre}Prerequisites}
%\begin{itemize}
%%\item StochSS 1.4 (or later) installed on your computer (please follow download and installation instructions at \url{www.stochss.org}). 
%%\item A basic understanding of well-mixed discrete stochastic simulations and models based on ordinary differential equations \cite{dan,sundials}.
%\item A basic knowledge of the mesoscopic reaction-diffusion master equation and of the Next Subvolume Method (NSM) \cite{nsm}.
%%\item  A basic knowledge of the functionalities in the StochSS GUI (please consult the \textit{Basic Introduction to StochSS} tutorial).
%%\item The following login screen appears in your browser; please log in.
%\end{itemize}

%\begin{figure}[!ht]
%\centering
%\includegraphics[scale=0.55]{T4/user-login.pdf}
%\end{figure}

%\newpage

The spatial stochastic simulation capabilities in StochSS are based on PyURDME \cite{urdme}. PyURDME is a general software framework for modeling and simulation of stochastic reaction-diffusion processes on unstructured, tetrahedral (3D) and triangular (2D) meshes. The current core simulation algorithm is based on the mesoscopic reaction-diffusion master equation (RDME) model. The default solver is an efficient implementation of the next subvolume method (NSM) \cite{nsm}.
%
%\section{Spatial Stochastic Simulations}

\section{An Instructive Example}
We will build a simple model based on an annihilation reaction between two chemical species generated  by sources located at the two opposite bases of an enclosing cylinder.
%The creation of our new spatial model using the \textbf{Model editor} progresses through the following steps:
\begin{itemize}
\item Navigate to the main \textbf{Model editor}.
\item Add a new model. Select \textit{Population, spatial} in the dropdown menu.
 \item Click \textbf{Mesh} and select \textit{Cylinder}. You will notice that the cylindrical mesh is divided into three subdomains. Species $A$ should be generated in subdomain 1 and species $B$ in subdomain 3.
\item Add two species, $A$ and $B$, both with diffusion constant 1.
\item Click \textbf{Initial Condition}, select \textit{scatter}, and add 500 molecules of species $A$ in subdomain 1 and 500 molecules of species $B$ in subdomain 3. 
\item Add two parameters, $k0$ and $k1$, and set their values to 1 and 100, respectively.
\item Add three reactions:%, $R1$, $R2$, and $R3$: $\emptyset\overset{k_1}{\rightarrow} A$, $\emptyset\overset{k_1}{\rightarrow} B$, and $A+B\overset{k_0}{\rightarrow}\emptyset$.
\begin{align*}
\textrm{R1}:&\quad \emptyset\overset{k_1}{\rightarrow} A\\
\textrm{R2}:&\quad \emptyset\overset{k_1}{\rightarrow} B\\
\textrm{R3}:&\quad A+B\overset{k_0}{\rightarrow}\emptyset
\end{align*}
\item Reaction $R1$ should be restricted to subdomain 1 and reaction $R2$ to subdomain 3. Reaction $R3$ should be allowed throughout the whole domain.
% \item Click on the \textcolor{blue}{`Species'} tag to define chemical species names and their diffusion coefficient. 
% \item Use the \textcolor{blue}{`Add species'} button on the right to add \textit{A} and \textit{B} species both with diffusion coefficient $D=1$.
% \item Click on the \textcolor{blue}{`Parameters'} tag to define model parameters names and their expressions. 
% \item Use the \textcolor{blue}{`Add parameter'} button to add parameter \textit{k1} with a value of $100$.
% \item Click on the \textcolor{blue}{`Reactions'} tag to define reaction names, reactants, products and propensities.
% \item Use the \textcolor{blue}{`Add reaction'} button to add reactions \textit{R1, R2, R3} (see Fig.~\ref{fig:2}). To insert reactions R1 and R2 (`birth' reactions) leave the `Reactants' text box empty and type in the name of the generated species in the `Products' text box.
% To define R3 (the annihilation reaction), enter the species names separated by a comma in the 
%`Reactants' text box. Leave the `Products' text box empty. Check out the on-screen help buttons.

 
 %Select the proper buttons in sections 3 and 4 on the Mesh page. Keep in mind that species A is generated in subdomain 1, whereas species B is generated in subdomain 3. Both species are allowed to diffuse everywhere within the cylinder (in all three subdomains). In other words, reaction R1 should be limited to subdomain 1 and reaction R2 to subdomain 3. Reaction R3 can happen in all three subdomains.
 %\item \textbf{Click} on the \textcolor{blue}{`Add Initial Condition'} button and \textcolor{blue}{`Select'} `Scatter`, `A`, `Subdomain = 1`,`Count = 500`. \textbf{Click} ', `B`, `Subdomain = 3',`Count = 500'. \textbf{Click} on the \textcolor{blue}{`Add Initial Condition'} button.
\item The model is now complete and ready to be simulated.

\item Navigate to the \textbf{Simulation manager} page.

\item Select the spatial model you just created and click \textbf{Next}.
\item Setup your simulation parameters: name, time, data storage frequency and realizations. 
\item You can specify a random seed for the random number generator under \textbf{Advanced Settings}.
%\item To define the initial random seed, select \textbf{Advanced Settings}.
\item Click \textbf{Run locally}.
\item In a few seconds you will be directed to the \textbf{Job Status} page where you can check the status of your simulation.
\item Once your simulation is complete, click \textbf{View results} to open the \textbf{Job summary} page, where you can visualize the diffusion of the two species over time within the cylindrical container and download the output files of the simulation.

\end{itemize}

%\begin{figure}[!ht]
%\centering
%\includegraphics[scale=0.64]{T4/reactions-spatial.pdf}
%\caption{`Reactions' page}
%\label{fig:2}
%\end{figure}

%\newpage

