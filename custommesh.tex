%!TEX root = main.tex

\chapter{Import a custom mesh}

When setting up a spatial problem in StochSS information about the geometry needs to be provided through two files: a
file containing the mesh information in DOLFIN XML format and a text file containing information about subdomains. The
file with subdomain information is optional in general (if it is not provided the entire mesh will belong to one subdomain).
Below is a tutorial on how to provide that information in a variety of cases.

\section{Mesh library}

The first and simplest scenario is using one of the default built-in meshes provided by StochSS in the Mesh Library.
These include simple example mesh and subdomain combinations that are common in biological modeling. For example
there is a unit sphere with a membrane (i.e. the volume is one subdomain and the surface is another), a unit cube witha
membrane, a cylinder with the two ends being different subdomains among others. If for a given problem these are not
sufficient then mesh and subdomain information can be provided by attaching files. One simple case where creating and
attaching a file may be necessary could involve using the DOLFIN built in meshes but with different subdomains than
provided in the Mesh Library. This can be done as follows.

\section{Using DOLFIN Built-In Geometry with Subdomains Outside of Mesh Library}

For example consider a sphere with three different subdomains: the top half of the membrane, the bottom half of the
membrane and the volume. This can be done using PyURDME as follows (PyURDME and FEniCS are software packages
that are automatically installed during the StochSS installation and DOLFIN is a library within FEniCS).

In an IPython Notebook, import the appropriate libraries:

\begin{ipythonnb}
import os,pyurdme,dolfin
\end{ipythonnb}

Next classes are constructed to define the desired subdomains:

\begin{ipythonnb}
class top_half_membrane(dolfin.SubDomain):
	def inside(self,x,on_boundary):
	return on_boundary and x[2]>=0
	
class bottom_half_membrane(dolfin.SubDomain):
	def inside(self,x,on_boundary):
	return on_boundary and x[2]<=0
\end{ipythonnb}

Then a pyurdme model is created with the desired geometry and marked subdomains:

\begin{ipythonnb}
class sphere_top_bot(pyurdme.URDMEModel):
	def __init__(self,model_name="polarization"):
		pyurdme.URDMEModel.__init__(self,model_name)
		
		#DefineGeometry
		sphere=dolfin.Sphere(dolfin.Point(0.0,0.0,0.0),1.0)
		self.mesh=pyurdme.URDMEMesh(mesh=dolfin.Mesh(sphere,10))

		cell_function=dolfin.CellFunction("size_t",self.mesh)
		cell_function.set_all(1)

		facet_function=dolfin.FacetFunction("size_t",self.mesh)
		facet_function.set_all(0)

		top_membrane=top_half_membrane()
		top_membrane.mark(facet_function,2)

		bottom_membrane=bottom_half_membrane()
		bottom_membrane.mark(facet_function,3)

		self.add_subdomain(cell_function)
		self.add_subdomain(facet_function)

		#Define initial populations
		self.set_initial_condition_scatter({},[1])
\end{ipythonnb}

\begin{ipythonnb}
model = sphere_top_bot()
\end{ipythonnb}

The subdmain infomation can then be extracted from the model as follows:
\begin{ipythonnb}
sd = model.get_subdomain_vector()
\end{ipythonnb}

And finally two files are created to store the mesh and subdomain information respectively in the the correct format:
\begin{ipythonnb}
\# This is a dolfin xml file with mesh information
dolfin.File("sphere_top_bot_mesh.xml") << model.mesh
\end{ipythonnb}

\begin{ipythonnb}
\ampersand This is the file to capture information about subdomains
with open("sphere_top_bot_subdomains.txt",'w') as fd:
	for ndx,val in enumerate(sd):
		fd.write("{0},{1}\n".format(ndx,val))
\end{ipythonnb}
Now that these two files have been created they can be attached in the Mesh tab of the model editor in StochSS and the
rest of the problem can be defined from there. Again it is important to note that if a problem does not involve subdomains
then only the mesh XML file is necessary and the entire mesh will be given the same domain! One last situation that could
arise in defining spatial information is that a mesh has to be created in external software then converted to the correct
format.

\section{Advanced mesh generation}

For many real world applications the geometry involved will be more complicated than the built?in meshes provided by
DOFLIN (and by extension StochSS). In these situations a mesh will have to be created in an external program such as
Gmsh then converted into the DOLFIN XML format. Once the mesh has been created in an external program the
conversion to DOLFIN XML format can easily be done by the built in script dolfin?convert [1]. The following command line
code demonstrates how to convert from the Gmsh format (suffix .msh or .gmsh) to DOLFIN XML format.

The following needs to be run from the command line in the Terminal:
\begin{verbatim}
dolfin-convert mesh.msh mesh.xml
\end{verbatim}

File formats that are currently supported by the dolfin-convert script can be found in the table below:

\begin{table}[htp]
   \begin{tabular}{| l | l |} % Column formatting, @{} suppresses leading/trailing space
   \toprule
      Suffix   & File Format\\
      \midrule
     .xml & DOLFIN XML format\\
     .ele / .node & Triangle file format\\
     .mesh & Medit format, generated by TetGen with option ?g\\
     .msh / .gmsh & Gmsh version 2.0 format\\
     .grid & Diffpack tetrahedral grid format\\
     .inp & Diffpack tetrahedral grid format\\
     .e / .exo & Sandia Exodus II file format\\
     .ncdf & ncdump'ed Exodus II file format\\
     .vrt / .cell & StarCD tetrahedral grid format\\
     \bottomrule
   \end{tabular}
   \end{table}



The XML file that is created from the dolfin?convert command can then be used directly in StochSS. If subdomains are
required then a text file can be created using code similar to that above. Consider the situation where the XML file created
was called "coli.xml" then the code to create a subdomain file consisting of the outer membrane would look as follows:

\begin{ipythonnb}
import os,pyurdme,dolfin
\end{ipythonnb}

\begin{ipythonnb}
class Membrane(dolfin.SubDomain):
	def inside(self,x,on_boundary):
	return on_boundary
\end{ipythonnb}

\begin{ipythonnb}
class ecoli(pyurdme.URDMEModel):
	def __init__(self,model_name="polarization"):
		pyurdme.URDMEModel.__init__(self,model_name)
		
		#DefineGeometry
		self.mesh=pyurdme.URDMEMesh(mesh=dolfin.Mesh('coli.xml'))

		cell_function=dolfin.CellFunction("size_t",self.mesh)
		cell_function.set_all(1)

		facet_function=dolfin.FacetFunction("size_t",self.mesh)
		facet_function.set_all(0)

		membrane=Membrane()
		membrane.mark(facet_function,2)

		self.add_subdomain(cell_function)
		self.add_subdomain(facet_function)

		#Define initial populations
		self.set_initial_condition_scatter({},[1])
\end{ipythonnb}

\begin{ipythonnb}
model=ecoli()
\end{ipythonnb}

\begin{ipythonnb}
sd=model.get_subdomain_vector()
\end{ipythonnb}

\begin{ipythonnb}
\#This is the file to capture information about subdomains
with open("coli_subdomains.txt",'w') as fd:
	for ndx,val in enumerate(sd):
		fd.write("{0},{1}\n".format(ndx,val))
\end{ipythonnb}

Now all of the necessary files have been created. The "coli.xml" file that was created using a dolfin?convert command and
the "coli\_subdomains.txt" just created can be uploaded directly to StochSS and used accordingly.

\section{References}

1. A. Logg, K.?A. Mardal, G. N. Wells et al. Automated Solution of Partial Differential Equations by the Finite Element Method: The FEniCS Book. Springer 2012.

